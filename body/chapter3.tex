\section{具身智能的物理对抗攻击算法}
\subsection{算法概述}
近期在机器人学领域,视觉-语言-动作(VLA)模型作为一种变革性方法崭露头角,通过将视觉与语言输入整合至端到端学习框架,使机器人能够执行复杂任务。尽管VLA模型展现出强大能力,却也引入了新的攻击面,使其易受对抗性攻击影响。然而,由于从数字世界过渡到物理世界的挑战,当前的攻击方法往往缺乏实用性,而现有的用于目标检测的物理攻击未能实现多视图有效性和自然性。

为了解决这个问题,本文提出了一种优化物体纹理信息的对抗攻击方法,通过将具有可学习纹理3D物体渲染生成的对抗补丁附加到物体对象上
\subsection{问题定义}

给定图像集合$\mathcal{I} $、指令集合$\mathcal{L} $、3D物体$o$、物体纹理特征$t$、相机视角集合$\mathcal{V}$ 和VLA模型$F: x\rightarrow  y$ ,我们的目标是为场景中的目标物体生成透明对抗纹理$t_{adv}$,以使这些物体导致VLA模型执行失败,并采取特定的被攻击行动。该问题可表述为寻找使VLA模型采取特定动作损失最小化的最优纹理。

具体来说,对于给定的VLA模型$F$,其输入$x$包含视觉信息$I \in \mathcal{I}$和语言指令$L \in \mathcal{L}$,输出为动作序列$y$。我们的目标是生成一个对抗纹理$t_{adv}$,使得当将其应用到3D物体$o$上时,从任意视角$v \in \mathcal{V}$渲染得到的图像$\hat{I} = \mathfrak{R}(o, t_{adv}; v)$与原始图像$I$组合后,能够误导VLA模型$F$产生错误的动作预测。

该问题可形式化为以下优化问题:

\begin{equation}
    \begin{aligned}
    \min_{t_{adv}} &\quad \left\{\mathbb{E}_{v \thicksim  \mathcal{V} } \mathfrak{L} _{adv}\left[ F (\mathfrak{R}  (o, t_{adv}; v )+\mathcal{I}, \mathcal{L} ), y \right ] \right\},\\
    \mathrm{s.t.} &\quad t_{adv} \in [0,1]^m,
    \end{aligned}
\end{equation}
其中,$\mathfrak{L}$ 表示对抗性损失,$\hat {I}=\mathfrak{R} (o, t_{adv})$表示渲染得到的图像,$t_{adv}$表示生成的待优化对抗纹理,其值域被约束在$[0,1]^m$范围内,$y$ 表示真值动作标签,$\hat{x}+\mathcal{I}$ 表示应用了对抗纹理渲染图象的输入信息,$[0,1]$ 表示有效的像素值范围,$m$ 表示纹理特征的维度,$\mathcal{V}$ 表示可能的相机视角集合,$y$ 表示期望模型输出的目标动作序列。

该优化问题的目标是找到一个最优的对抗纹理$t_{adv}$,使得从任意视角渲染得到的图像都能有效地误导VLA模型,同时保持纹理的自然性和物理可实现性。

\subsection{相机视角集合}

从原始数据中给定不同时间状态下的同一物体的位姿集合$\mathcal{T}$、物体的四元数集合$\mathcal{Q}$、相机内参矩阵$A$和相机外参矩阵集合$\mathcal{E}$,我们可以构建相机视角集合$\mathcal{V}$。具体而言,对于每个时间步$i$,我们有以下定义:

\begin{itemize}
    \item 相机内参矩阵$A \in \mathbb{R}^{3 \times 3}$,包含相机的焦距、主点等参数;
    \item 相机外参矩阵集合$\mathcal{E} = \{E_1, E_2, ..., E_n\}$,其中每个外参矩阵$E_i \in \mathbb{R}^{4 \times 4}$表示相机在世界坐标系中齐次坐标变换矩阵,表示相机在世界坐标系中的位置和旋转状态;
    \item 物体位姿集合$\mathcal{T} = \{T_1, T_2, ..., T_n\}$,其中每个位姿$T_i \in \mathbb{R}^{4 \times 4}$表示物体在世界坐标系中的位置;
    \item 物体四元数集合$\mathcal{Q} = \{q_1, q_2, ..., q_n\}$,其中每个四元数$q_i \in \mathbb{R}^{1 \times 4}$表示物体的旋转状态和方向。
\end{itemize}
 
基于这些参数,我们可以构建相机视角集合$\mathcal{V}$:

\begin{equation}
    \mathcal{V} = \{v_i = (A, E_i, T_i, q_i) | i = 1,2,\dots,n\}.
\end{equation}
% 其中,每个视角$v \in \mathcal{V}$包含以下信息:
% \begin{itemize}
%     \item 相机内参矩阵$A$:定义相机的内部参数,包括焦距$(f_x, f_y)$和主点$(c_x, c_y)$
%     \item 相机外参矩阵$E_i$:定义相机在世界坐标系中的位置和方向
%     \item 物体位姿$T_i$:定义物体在世界坐标系中的位置和方向
%     \item 物体四元数$q_i$:定义物体的旋转状态
% \end{itemize}

% 对于每个视角$v$,我们可以通过以下步骤计算投影矩阵$P_v$:

% \begin{equation}
%     P_v = A \cdot E_i \cdot T_i^{-1}
% \end{equation}

% 这个投影矩阵将用于3D物体渲染过程,将3D点投影到2D图像平面上。通过这种方式,我们确保了在不同视角下都能正确渲染物体的外观,这对于生成具有视角不变性的对抗纹理至关重要。

\subsection{针对VLA模型的物理对抗攻击框架}

本文提出的物理对抗攻击框架主要包含三个核心组件:3D物体渲染模块、对抗纹理优化模块和物理约束模块。该框架的整体架构如图\ref{fig:framework}所示。

\subsubsection{3D物体渲染}

为解决3D物体的纹理特征在多视角下的都具有攻击的有效性,本文采用基于物理的可微分渲染器对物体纹理进行优化。首先,从原始数据集获取物体的位姿、物体的四元数和相机内外参数,随后通过3D物体渲染模块将带有对抗纹理的3D物体从不同视角渲染成2D图像,再利用多视角梯度,通过基于梯度的方法优化贴片纹理。具体来说,对于给定的3D物体$o$和对抗纹理$t_{adv}$,我们使用可微分的渲染器$\mathfrak{R}$从视角$v$生成图像$\hat{I}$: 
\begin{align}
    I &= \mathfrak{R}(o, t; v),\\
    \hat{I} &= \mathfrak{R}(o, t_{adv}; v),
\end{align}
其中,$I$和$\hat{I}$分别表示未使用对抗纹理和使用对抗纹理后的图像。

\subsubsection{对抗纹理优化模块}
对抗纹理优化模块是框架的核心,负责生成最优的对抗纹理。该模块采用基于梯度的优化方法,通过最小化对抗损失函数来更新纹理参数:
\begin{equation}
    t_{adv}^{t+1} = t_{adv}^t - \alpha \nabla_{t_{adv}} \mathfrak{L}_{adv}
\end{equation}

其中,$\alpha$为学习率,$\nabla_{t_{adv}} \mathfrak{L}_{adv}$表示对抗损失对纹理参数的梯度。为了确保生成的纹理具有自然性,我们在优化过程中引入了感知损失和风格损失:
\begin{equation}
    \mathfrak{L}_{total} = \mathfrak{L}_{adv} + \lambda_1\mathfrak{L}_{perceptual} + \lambda_2\mathfrak{L}_{style}
\end{equation}

\subsubsection{物理约束模块}
物理约束模块确保生成的对抗纹理在物理世界中是可实现的。该模块主要考虑以下约束:

\begin{itemize}
    \item \textbf{视角不变性}:确保从不同视角观察时,对抗纹理都能有效误导VLA模型
    \item \textbf{光照鲁棒性}:保证在不同光照条件下,对抗纹理的效果保持稳定
    \item \textbf{打印可行性}:确保生成的纹理可以通过实际打印设备制作
    \item \textbf{材质兼容性}:考虑不同材质表面的纹理附着特性
\end{itemize}

这些约束通过以下方式实现:

\begin{equation}
    \begin{aligned}
    \min_{t_{adv}} &\quad \mathfrak{L}_{total}\\
    \mathrm{s.t.} &\quad \|t_{adv} - t_{adv}^{print}\|_2 \leq \epsilon\\
    &\quad \|t_{adv}^{v_1} - t_{adv}^{v_2}\|_2 \leq \delta, \forall v_1,v_2 \in \mathcal{V}
    \end{aligned}
\end{equation}

其中,$t_{adv}^{print}$表示可打印的纹理版本,$\epsilon$和$\delta$为约束阈值。

\subsubsection{实现细节}
在实际实现中,我们采用以下策略来提升框架的性能:

\begin{enumerate}
    \item 使用预训练的VLA模型作为攻击目标,确保攻击的通用性
    \item 采用多尺度优化策略,从粗到细逐步优化纹理细节
    \item 实现并行渲染和优化,提高计算效率
    \item 引入自适应学习率调整机制,提高优化稳定性
\end{enumerate}

